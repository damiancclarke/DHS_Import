\documentclass{article}[11pt,subeqn]

\usepackage{fancyhdr}
\pagestyle{fancy}
\usepackage[pdftex]{graphicx}
\usepackage{setspace}
\usepackage{framed}
\usepackage{lastpage}
\usepackage{amsmath}
\usepackage{hyperref}
\usepackage[utf8]{inputenc}

\title{Automatically Importing, Appending and Merging the Demographic and Health Surveys}
\author{Damian C. Clarke\thanks{Contact mail \href{mailto:damian.clarke@economics.ox.ac.uk}{damian.clarke@economics.ox.ac.uk}}}
\date{\today}

  \setlength\topmargin{-0.375in}
%  \setlength\headheight{20pt}
 \setlength\headwidth{5.8in}
  \setlength\textheight{8.9in}
  \setlength\textwidth{5.8in}
  \setlength\oddsidemargin{0.3in}
%  \setlength\evensidemargin{-0.5in}
%  \setlength\parindent{0.25in}
  \setlength\parskip{0.25in}

\usepackage{natbib}
\bibliographystyle{abbrvnat}
\bibpunct{(}{)}{;}{a}{,}{,}

\usepackage{lscape}
\usepackage{rotating}
\usepackage{multirow}
\usepackage{rotating,capt-of}
\usepackage{array}

\usepackage[update,prepend]{epstopdf}

\usepackage[font=sc]{caption}

%NEW COMMANDS
\newcommand{\Lagr}{\mathcal{L}}
\newcommand{\vect}[1]{\mathbf{#1}}
\newcolumntype{P}[1]{>{\raggedright}p{#1\linewidth}}

\usepackage{appendix}
\usepackage{booktabs}
%\usepackage{cleveref}

\fancyhead{}
\fancyfoot{}
\fancyhead[L]{\textsc{DHS Use}}
\fancyhead[R]{\textsc{Damian C. Clarke}}
\fancyfoot[C]{\textsc{\thepage \ of  \pageref{LastPage}}}

\usepackage{pgf,tikz}
\usepackage{pgfplots}
\usetikzlibrary{decorations}
\usetikzlibrary{shapes,arrows}

\begin{document}
\begin{spacing}{1.25}

\maketitle

\begin{abstract}
I provide here a quick guide to producing multi-country DHS files from the 1500+ surveys provided on-%
line.  This guide is provided as a companion to the program DHS\_Import.py and DHS\_Multicountry.do
which completely automate the creation of a comprehensive database of all publicly-available DHS files.
In order to create these databases the user simply requires an internet connection, and the ability to
run the two provided files; the first using the free program Python and the second using the statistical
program Stata. The programs provided are written to include all surveys available at the date of use, 
even if those surveys were not available when the program was written.
\end{abstract}

\section{Introduction}
The Demographic and Health Surveys (DHS) are a set of surveys collected in developing countries which
provide data relating to maternal health and fertility, child health, as well as specific modules on
HIV, anemia and maternal mortality (among other variables).  These files are publicly available on-line
at \href{http://www.measuredhs.com/}{http://www.measuredhs.com}, along with a full description of their 
contents, methodology and use.

Currently the DHS files are stored by survey country and year, resulting in approximately 1600 separate
files over the period 1990-2011.  Hence, a user interested in working with cross-country data, must either
manually `point and click' to download these and then merge the downloaded files, or write a script to
download and unzip the files and then merge the resulting files.  I provide here such a routine which 
allows for the automatic importing, appending and merging of all publicly available DHS datasets. This
routine consists of two files, a file written using the computer language Python which downloads, unzips 
and saves all files, and then a stata do file to process the DHS files.  The reason I write the downloading 
script in Pyton rather than the more commonly used (by economists) ado language for Stata is due to 
Python's flexibility, as well as its simplicity and the fact that it is free and installed on many operating 
systems `out of the box'\footnote{For example Python is included by default on ubuntu.  Windows users 
without Python installed can install version 2.7.3 (suggested) \href{http://www.python.org/getit/}{here}.}.  
In order to use this program it is not necessary for the user to know any Python, although a familiarity 
with Stata is required as all resulting datasets are produced in Stata's \texttt{.dta} format.


\section{Files}
\label{scn:files}
Figure \ref{fig:DHS} provides a description of the DHS datasets including optimal merging direction.
There are seven distinct datasets provided by the DHS (excluding the Wealth Index data and Height
and Weight data).  Each of these datasets merges directly with the Individual Women's Recode (IR),
with the exception of the Male Recode (MR) which only merges to the IR in the case that a male
and female live together and both report sharing a relationship.  In this case, the merged database is
available as the couple's recode (CR).

\tikzstyle{decision} = [diamond, draw, fill=blue!20, 
    text width=4.5em, text badly centered, node distance=3cm, inner sep=0pt]
\tikzstyle{block} = [rectangle, draw, fill=blue!20, 
    text width=5em, text centered, rounded corners, minimum height=4em]
\tikzstyle{line} = [draw, -latex']
\tikzstyle{line1} = [draw]
\tikzstyle{cloud} = [draw, ellipse,fill=red!20, node distance=3cm,
    minimum height=2em]

\begin{figure}[ht!]
\begin{center}
\caption{Links between DHS Recodes}
\label{fig:DHS}
\begin{tikzpicture}[node distance = 3cm, auto]
    % Place nodes
    \node [block] (HR) {Household Recode (HR)};
    \node [block, below of=HR] (PR) {HH Member Recode (PR)};
    \node [block, below of=PR] (CR) {Couple's Recode (CR)};
    \node [block, left of=CR, node distance=4cm] (IR) {Individual Women's Recode (IR)};
    \node [block, right of=CR, node distance=4cm] (MR) {Male Recode (MR)};    
    \node [block, below of=IR] (KR) {Children's Recode (KR)};
    \node [block, below of=CR] (BR) {Birth Recode (BR)};
    \node [below of=CR, node distance=1.2cm] (link) {};    
    \node [below of=CR, node distance=30.5pt] (link1) {};    

    % Draw edges
    \path [line] (HR) --node [right] {\textbf{1:m}} (PR);
    \path [line] (PR) -| node [above]{\textbf{1:m}}(IR);
    \path [line] (PR) -| node [above]{\textbf{1:m}}(MR);
    \path [line,dashed] (IR) -- node [above]{\textbf{1:m}}(CR);
    \path [line,dashed] (MR) -- node [above]{\textbf{1:m}}(CR);    
    \path [line] (IR) --node [right] {\textbf{1:m}} (KR);
    \path [line1] (IR) |- (link);
    \hspace{-3.5pt}\path [line] (link1) -- node [right] {\textbf{1:m}}(BR);
    
    \end{tikzpicture}
\end{center}
\end{figure}

Each file link is joined by a unique merge code and the ``DHS guarantees that their files can be matched 
seamlessly whenever a relationship is possible''.  The DHS provides a description of how to join each 
mergeable set of files on its website\footnote{\url{http://www.measuredhs.com/data/Merging-Datasets.cfm}},
however this table assumes that the user is working with data from a single country.  Hence, to ensure
that data still merges seamlessly, I add two new variables to each file; \_cou (country) and \_year 
(year) (see lines 27--29 of the file DHS\_multicountry.do).  In this way, files are merged as described in 
table \ref{tab:merge}.

\begin{table}[htpb!]
\begin{center}
\caption{DHS Merge variables}
\label{tab:merge}
\begin{tabular}{lll} 
\toprule
& \multicolumn{2}{l}{Secondary Files} \\ \cmidrule{2-3}
& Match for household & Match for women \\
\midrule
Base & \_cou + \_year + HV001 + HV002 & \_cou + \_year + V001 + V002 + V003 \\
\begin{footnotesize} \end{footnotesize} & \begin{footnotesize} \end{footnotesize} & \begin{footnotesize} \end{footnotesize} \\
Women & \_cou + \_year + V001 + V002 &  \\
\begin{footnotesize} \end{footnotesize} & \begin{footnotesize} \end{footnotesize} & \begin{footnotesize} \end{footnotesize} \\
Children & \_cou + \_year + V001 + V002 & \_cou + \_year + V001 + V002 + V003 \\
\begin{footnotesize} \end{footnotesize} & \begin{footnotesize} \end{footnotesize} & \begin{footnotesize} \end{footnotesize} \\
Men & \_cou + \_year + MV001 + MV002 & Couples:  \_cou + \_year + \\
 & &  MV001 + MV002 + MV034i\\
\bottomrule
%\multicolumn{3}{l}{Agregar una nota acá.}
\end{tabular}
\end{center}
\end{table}

Given that DHS files exist for each recode type (figure \ref{fig:DHS}) and each country/year pair available, 
the file DHS\_multicountry.do creates a cross-country dataset for each file type by merging each recode of a 
given type together.  Thus, the resulting files will be HR\_World, PR\_World, IR\_World and so on.  Given that 
each of the mergeable `World' files still has a unique merge code (as described in table \ref{tab:merge}), 
these can then be merged according to the user's needs.  Using all files available at the date this document was 
produced, the size of the resulting files are listed in table \ref{tab:files}.  Whilst the sizes of these datasets
is relatively large for work on a personal computer, these can easily be shrunk to a manageable size by either
limiting the quantity of variables included, or by splitting the dataset into various smaller datasets.  The
datasets listed in table \ref{tab:files} include all cross-country variables, many of which will not be relevant
in any given analysis.  Due to this fact, the user can specify in the program which variables they wish to keep
before appending the individual country files (line 49 of DHS\_Multicountry.do).

\begin{table}[htpb!]
\begin{center}
\caption{DHS Merge variables}
\label{tab:files}
\begin{tabular}{ll} 
\toprule
File Name & Size \\
\midrule
HR\_World & 21.4 GB \\
PR\_World & 9.2 GB \\
IR\_World & 16.9 GB \\
%CR\_World &  \\
MR\_World & 1.2 GB \\
KR\_World & 2.9 GB \\
BR\_World & 9.5 GB \\
\bottomrule
%\multicolumn{3}{l}{Agregar una nota acá.}
\end{tabular}
\end{center}
\end{table}

\section{Creating the Cross-Country DHS Files}
\subsection{Running the Programs}
The programs provided along with this document; DHS\_Import.py and DHS\_Multicountry.do, are operating-system
independent, and require only that the user possess access to Python (freely available) and Stata (available at most 
universities).  These programs have been successfully tested on Ubuntu 12.04 and Windows 7, and should work
without difficulity on other Linux and Windows systems, plus MacOS. To run the programs, the user must download
the two program files, plus an auxiliary text file (DHS\_Countries.txt) which lists the full set of DHS survey countries.
These three files should be saved in the same working directory.

In order to import and build the DHS data, the user should first run the Python file to download and unzip all datasets. 
This is done as follows (assuming that the user has already installed Python): firstly download the file from \url{https://sites.google.com/site/damiancclarke/computation} and save in any directory, also ensure that you are 
logged into the DHS data download \href{http://www.measuredhs.com/data/available-datasets.cfm}{website} 
using your user name and password (available via on-line application).  Secondly, open the command line 
(also known as the terminal) of your computer\footnote{In ubuntu this can be accessed by simply typing 
\texttt{Ctrl + Alt + T}, in Windows by typing \texttt{cmd} at the Run option of the Start menu, and in Mac 
via the Applications folder (\texttt{Applications$\rightarrow$Utilities$\rightarrow$Terminal})}.  Finally,
at the command line change to the working directory where you saved DHS\_Import.py (using the command 
\texttt{cd}), and enter the following:

\texttt{python DHS\_Import.py file1 file2}

\noindent where the argument \texttt{file1} represents the directory where you want to download the unzipped 
\texttt{.dta} files, and \texttt{file2} the folder where your operating system by default downloads files from 
the web (this may be something like ``$\sim$\slash Downloads'' on Linux-based operating systems or 
``C\slash Downloads'' on Windows).  Once having entered this command the program will proceed to download
all available datasets, and save them in the directory specified \texttt{file1}.  Given that the speed at which
the program runs depends upon the strength of the internet connection, it is recommended at this stage to have
access to a stable connection.  It is also highly recommended that download settings be changed (if necessary) to 
ensure that the user need not authorize each individual download in a pop-up dialog box.

Once having downloaded all current DHS files, these are appended together into a final set of usable databases
using the Stata file DHS\_Multicountry.do.  This file requires very few changes: the user simply need specify
the values for the global and local variables on lines 14-16.  The first global variable should be set to the directory
in which the Python program was saved, the second global to the argument \texttt{file1} from the Python command,
(the location of the data) and finally the local ``clean'' must be specified.  For this local the user should enter
\texttt{yes} if they wish for files added by the program to be removed upon completion (this is advisable where
the user is concerned about the use of hard disk space on their computer).

\subsection{Resulting Files}
These programs will result in a full set of DHS databases as described in table \ref{tab:files}, along with 
three raw text files.  The first of these text files contains one line for each country (and country code) and 
year that the survey is run (eg CO Colombia 1990).  The second lists all individual databases downloaded (for 
each country/year pair multiple surveys exist), and finally, a similar file is produced for use by the Stata
\texttt{.do} file.  These files may be of interest to the user should they require summary statistics of years
and places surveyed in the DHS.

\subsection{Final Points}
These programs are provided for open use and can be modified in any way that the user desires.  Any comments, 
queries or suggestions are welcome.  The Stata \texttt{.do} file provided here does not include commands to 
merge datasets once they have been produced, as the types of merges required and the variables included will 
vary widely depending upon the interest of the user.  I do however have \texttt{.do} files which program these 
merges, and am happy to make these available to interested users.  For these files, or for any other details 
regarding these programs, please contact the author at
\href{mailto:damian.clarke@economics.ox.ac.uk}{damian.clarke@economics.ox.ac.uk}.  




\end{spacing}
\end{document}